\PassOptionsToPackage{unicode=true}{hyperref} % options for packages loaded elsewhere
\PassOptionsToPackage{hyphens}{url}
\PassOptionsToPackage{dvipsnames,svgnames*,x11names*}{xcolor}
%
\documentclass[]{article}
\usepackage{lmodern}
\usepackage{graphicx}
\usepackage{ragged2e}
\usepackage{amssymb,amsmath}
\usepackage{ifxetex,ifluatex}
\usepackage{fixltx2e} % provides \textsubscript
\ifnum 0\ifxetex 1\fi\ifluatex 1\fi=0 % if pdftex
  \usepackage[T1]{fontenc}
  \usepackage[utf8]{inputenc}
  \usepackage{textcomp} % provides euro and other symbols
\else % if luatex or xelatex
  \usepackage{unicode-math}
  \defaultfontfeatures{Ligatures=TeX,Scale=MatchLowercase}
\fi
% use upquote if available, for straight quotes in verbatim environments
\IfFileExists{upquote.sty}{\usepackage{upquote}}{}
% use microtype if available
\IfFileExists{microtype.sty}{%
\usepackage[]{microtype}
\UseMicrotypeSet[protrusion]{basicmath} % disable protrusion for tt fonts
}{}
\IfFileExists{parskip.sty}{%
\usepackage{parskip}
}{% else
\setlength{\parindent}{0pt}
\setlength{\parskip}{6pt plus 2pt minus 1pt}
}
\usepackage{xcolor}
\usepackage{hyperref}
\hypersetup{
            pdftitle={Semana 10: Tries - Árvores de Prefixos},
            pdfauthor={Prof.~Dr.~Juliano Henrique Foleis},
            colorlinks=true,
            linkcolor=Maroon,
            filecolor=Maroon,
            citecolor=Blue,
            urlcolor=Blue,
            breaklinks=true}
\urlstyle{same}  % don't use monospace font for urls
\usepackage[margin=1in]{geometry}
\usepackage{graphicx,grffile}
\makeatletter
\def\maxwidth{\ifdim\Gin@nat@width>\linewidth\linewidth\else\Gin@nat@width\fi}
\def\maxheight{\ifdim\Gin@nat@height>\textheight\textheight\else\Gin@nat@height\fi}
\makeatother
% Scale images if necessary, so that they will not overflow the page
% margins by default, and it is still possible to overwrite the defaults
% using explicit options in \includegraphics[width, height, ...]{}
\setkeys{Gin}{width=\maxwidth,height=\maxheight,keepaspectratio}
\setlength{\emergencystretch}{3em}  % prevent overfull lines
\providecommand{\tightlist}{%
  \setlength{\itemsep}{0pt}\setlength{\parskip}{0pt}}
\setcounter{secnumdepth}{0}
% Redefines (sub)paragraphs to behave more like sections
\ifx\paragraph\undefined\else
\let\oldparagraph\paragraph
\renewcommand{\paragraph}[1]{\oldparagraph{#1}\mbox{}}
\fi
\ifx\subparagraph\undefined\else
\let\oldsubparagraph\subparagraph
\renewcommand{\subparagraph}[1]{\oldsubparagraph{#1}\mbox{}}
\fi

% set default figure placement to htbp
\makeatletter
\def\fps@figure{htbp}
\makeatother


\title{Semana 10: \emph{Tries} - Árvores de Prefixos}
\author{Prof.~Dr.~Juliano Henrique Foleis}
\date{}

\begin{document}
\maketitle

Estude com atenção os vídeos e as leituras sugeridas abaixo. Os
exercícios servem para ajudar na fixação do conteúdo e foram escolhidos
para complementar o material básico apresentado nos vídeos e nas
leituras. Quando o exercício pede que crie ou modifique algum algoritmo,
sugiro que implemente-o em linguagem C para ver funcionando na prática.
O único exercício que é necessário entregar está descrito na Seção
``Atividade Para Entregar''.

\hypertarget{vuxeddeos}{%
\section{Vídeos}\label{vuxeddeos}}

\href{https://youtu.be/Spmw4hTo7ek}{Tries: Árvores de Prefixos}

\hypertarget{leitura-sugerida}{%
\section{Leitura Sugerida}\label{leitura-sugerida}}

FEOFILOFF, Paulo. Estruturas de Dados. \emph{Tries (árvores digitais)}
\href{https://www.ime.usp.br/~pf/estruturas-de-dados/aulas/tries.html}{(Link)}

\hypertarget{exercuxedcios-dos-materiais-de-leitura-sugerida}{%
\subsection{Exercícios dos materiais de leitura
sugerida}\label{exercuxedcios-dos-materiais-de-leitura-sugerida}}

Exercícios 3.1, 3.2 da página do Prof.~Feofiloff (Tries (árvores
digitais)):
\href{https://www.ime.usp.br/~pf/estruturas-de-dados/aulas/tries.html}{(Link)}

\hypertarget{atividade-para-entregar}{%
\section{Atividade para Entregar}\label{atividade-para-entregar}}

A atividade a seguir é para ser feita individualmente e entregue via
Moodle no tópico da Semana 10. A data-limite para entrega é dia
16/10/2020 às 23:55. Em caso de cópia as atividades dos participantes
serão desconsideradas.

\hypertarget{descriuxe7uxe3o-da-atividade}{%
\subsection{Descrição da Atividade}\label{descriuxe7uxe3o-da-atividade}}

Clone (ou atualize!) o repositório da disciplina no
\href{https://github.com/julianofoleiss/BCC33A}{github}. A implementação
da trie está nos arquivos \emph{trie/asciitrie.c} e
\emph{trie/asciitrie.h}.

\textbf{1.} Implemente as funções a seguir:

\textbf{a.} Implemente as funções \emph{AT\_Criar}, \emph{AT\_Buscar},
\emph{AT\_Inserir} e \emph{AT\_Remover} conforme mostrado no
\href{https://youtu.be/Spmw4hTo7ek}{vídeo}.

\textbf{b.} Implemente a função \emph{AT\_Imprimir} de forma que produza
a saída mostrada no \href{https://youtu.be/Spmw4hTo7ek?t=4155}{vídeo}.

\textbf{c.} Escreva uma versão iterativa de \emph{AT\_Buscar} e
\emph{AT\_Inserir}.

\textbf{d.} Escreva uma versão iterativa de \emph{AT\_Remover}.
\textbf{DICA:} use uma estrutura de dados auxiliar para ``lembrar'' os
nós analisados durante o percurso na árvore.

\textbf{e.} Uma trie é dita limpa se cada uma de suas folhas corresponde
a uma chave. Em outras palavras, uma trie \emph{T} é limpa se para todo
nó \emph{N} que é folha de \emph{T}, então \emph{N--\textgreater{}estado
== ATE\_OCUPADO}. Escreva uma função \emph{AT\_Limpa} que retorna 1 se a
trie está limpa e 0, caso contrário.

\textbf{f.} Escreva uma função \emph{AT\_Tamanho} que retorna o número
de CHAVES armazenadas em uma trie. Sua implementação deve ser
preguiçosa, ou seja, não guarda nenhuma informação em relação ao tamanho
das sub-árvores na estrutura.

\textbf{g.} Reescreva a função \emph{AT\_Tamanho} usando uma
implementação ansiosa. Altere a estrutura \emph{ASCIITrie} de acordo.

\textbf{h.} Implemente a função \emph{AT\_Min}, que retorna a menor
chave da trie. Considere que a menor chave é aquela que é a primeira em
uma ordenação lexicográfica em relação as demais chaves. Por exemplo
``abc'' é lexicograficamente \emph{menor} que ``abd''.

\textbf{i.} Implemente a função \emph{AT\_Max}, que retorna a maior
chave da trie. Considere que a maior chave é aquela que é a última em
uma ordenação lexicográfica em relação as demais chaves. Por exemplo
``xyz'' é lexicograficamente \emph{maior} que ``xyy''.

\textbf{2.} Implemente a função \emph{int SubstringCountLenL(char * s,
int L)} que calcule o número de substrings distintas de comprimento L
que a string de entrada \emph{s} contém. Caso L seja maior que o
comprimento da string, retorne 0. Por exemplo, a string ``abcdef''
possui 4 substrings de tamanho 3 (abc, bcd, cde, def). \textbf{Dica:}
use uma trie para guardar as substrings.

\textbf{3.} Use os casos de teste fornecidos e teste sua implementação
do exercício 2. Preencha a tabela abaixo com o número de substrings
distintas em cada caso. \textbf{DICA:} Caso esteja dando estouro de
pilha (por conta de uma pilha de recursão muito grande), tente usar as
versões iterativas das funções de Inserção e Busca.

\begin{figure}
\centering
\includegraphics{tabela_sem10.png}
\caption{Resultados do Exercício 3}
\end{figure}

\hypertarget{vocuxea-deve-entregar}{%
\subsection{Você deve Entregar}\label{vocuxea-deve-entregar}}

Entregue em formato .zip os arquivos a seguir:

\begin{itemize}
\tightlist
\item
  Os arquivos-fonte desenvolvidos nos itens \textbf{1 e 2}, bem como os
  arquivos-fonte criados para realizar os testes. Faça um
  \emph{Makefile} para compilar o seu programa. Modularize conforme
  julgar necessário.
\item
  A Tabela preenchida no Exercício \textbf{3} em um \emph{pdf}.
\end{itemize}

\centering

\vspace{20pt}

\textbf{Por favor entregue como especificado acima!}

\vspace{50pt}

\textbf{A data-limite para entrega é dia 16/10/2020 às 23:55.}

\vspace{20pt}

\Large

\textbf{BONS ESTUDOS!}

\end{document}
