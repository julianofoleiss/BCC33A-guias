\PassOptionsToPackage{unicode=true}{hyperref} % options for packages loaded elsewhere
\PassOptionsToPackage{hyphens}{url}
\PassOptionsToPackage{dvipsnames,svgnames*,x11names*}{xcolor}
%
\documentclass[]{article}
\usepackage{lmodern}
\usepackage{graphicx}
\usepackage{ragged2e}
\usepackage{amssymb,amsmath}
\usepackage{ifxetex,ifluatex}
\usepackage{fixltx2e} % provides \textsubscript
\ifnum 0\ifxetex 1\fi\ifluatex 1\fi=0 % if pdftex
  \usepackage[T1]{fontenc}
  \usepackage[utf8]{inputenc}
  \usepackage{textcomp} % provides euro and other symbols
\else % if luatex or xelatex
  \usepackage{unicode-math}
  \defaultfontfeatures{Ligatures=TeX,Scale=MatchLowercase}
\fi
% use upquote if available, for straight quotes in verbatim environments
\IfFileExists{upquote.sty}{\usepackage{upquote}}{}
% use microtype if available
\IfFileExists{microtype.sty}{%
\usepackage[]{microtype}
\UseMicrotypeSet[protrusion]{basicmath} % disable protrusion for tt fonts
}{}
\IfFileExists{parskip.sty}{%
\usepackage{parskip}
}{% else
\setlength{\parindent}{0pt}
\setlength{\parskip}{6pt plus 2pt minus 1pt}
}
\usepackage{xcolor}
\usepackage{hyperref}
\hypersetup{
            pdftitle={Semana 13: Árvores B e B+},
            pdfauthor={Prof.~Dr.~Juliano Henrique Foleis},
            colorlinks=true,
            linkcolor=Maroon,
            filecolor=Maroon,
            citecolor=Blue,
            urlcolor=Blue,
            breaklinks=true}
\urlstyle{same}  % don't use monospace font for urls
\usepackage[margin=1in]{geometry}
\setlength{\emergencystretch}{3em}  % prevent overfull lines
\providecommand{\tightlist}{%
  \setlength{\itemsep}{0pt}\setlength{\parskip}{0pt}}
\setcounter{secnumdepth}{0}
% Redefines (sub)paragraphs to behave more like sections
\ifx\paragraph\undefined\else
\let\oldparagraph\paragraph
\renewcommand{\paragraph}[1]{\oldparagraph{#1}\mbox{}}
\fi
\ifx\subparagraph\undefined\else
\let\oldsubparagraph\subparagraph
\renewcommand{\subparagraph}[1]{\oldsubparagraph{#1}\mbox{}}
\fi

% set default figure placement to htbp
\makeatletter
\def\fps@figure{htbp}
\makeatother


\title{Semana 13: Árvores B e B+}
\author{Prof.~Dr.~Juliano Henrique Foleis}
\date{}

\begin{document}
\maketitle

Estude com atenção os vídeos e as leituras sugeridas abaixo.

\hypertarget{vuxeddeos}{%
\section{Vídeos}\label{vuxeddeos}}

\href{https://youtu.be/5mC6TmviBPE}{Árvores B}

\href{https://youtu.be/x-NNKmdHm94}{Árvores B+}

\hypertarget{leitura-sugerida}{%
\section{Leitura Sugerida}\label{leitura-sugerida}}

FEOFILOFF, Paulo. Estruturas de Dados. \emph{Árvores B}
\href{https://www.ime.usp.br/~pf/estruturas-de-dados/aulas/B-trees.html}{(Link)}

CORMEN, T. H; LEISERSON, C. E; RIVEST, R. L. e STEIN, C. Algoritmos:
Teoria e Prática. 3a edição. Ed. Elsevier. 2012. Capítulo 18 -- Árvores
B

SEDGEWICK, Robert e WAYNE, Kevin. Algorithms, 4th. ed.~Addison-Wesley,
2011. Capítulo 6, Seção \emph{B-Trees} (pp.~866-874) {[}Em Inglês{]}

\hypertarget{exercuxedcios-dos-materiais-de-leitura-sugerida}{%
\subsection{Exercícios dos materiais de leitura
sugerida}\label{exercuxedcios-dos-materiais-de-leitura-sugerida}}

Exercício 1.1 da página do Prof.~Feofiloff (Árvores B):
\href{https://www.ime.usp.br/~pf/estruturas-de-dados/aulas/B-trees.html}{(Link)}

\hypertarget{atividade-para-entregar}{%
\section{Atividade para Entregar}\label{atividade-para-entregar}}

Não há atividades para serem entregues essa semana.

\centering
\vspace{40pt}
\Large

\textbf{BONS ESTUDOS!}

\end{document}
